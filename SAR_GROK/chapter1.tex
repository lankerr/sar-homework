\chapter{SAR算法仿真}

\section{引言与研究背景}
在本章中,我们对合成孔径雷达(SAR)的正常仿真进行探讨,主要焦点是理想状态下的成像过程(无噪声环境)以及GPU在算法加速中的作用。SAR成像算法包括后向投影算法(BPA)、Chirp Scaling Algorithm(CSA)、Range Doppler Algorithm(RDA)和Omega-K Algorithm(wKA)。这些算法在不同场景下有各自的优势,其中BPA作为时域算法提供最高精度,但计算复杂度最高。通过GPU加速,我们可以显著提升其效率。

研究过程从算法数学模型推导开始,逐步实现代码,并对比CPU与GPU性能。最后,分析理想条件下成像结果,并讨论噪声缺失导致的特定现象。

\section{BPA算法的数学模型}
BPA是一种精确的时域成像算法,适用于任意运动轨迹和波形。考虑发射信号$s_{tx}(t)$,点目标$P(x,y,0)$在方位时间$\eta$的回波为:
\begin{equation}
s_r(t,\eta;P) = \sigma_P \cdot s_{tx}\left(t - \frac{2R(\eta;P)}{c}\right) \cdot \exp\left[-j\frac{4\pi}{\lambda}R(\eta;P)\right]
\end{equation}
其中,斜距历程$R(\eta;P) = \sqrt{(x - x_a(\eta))^2 + (y - y_a(\eta))^2 + R_0^2}$。

成像过程涉及对每个像素点累加补偿后的回波:
\begin{equation}
I(x,y) = \int s_r\left(\frac{2R(\eta;P)}{c}, \eta\right) \exp\left[j\frac{4\pi}{\lambda}R(\eta;P)\right] d\eta
\end{equation}

在理想无噪声条件下,点扩散函数(PSF)呈现清晰峰值,但由于相干积累完美,可能出现X形artifact。

\section{GPU加速探究}
BPA的计算复杂度为$O(N_a \times N_r \times N_{img}^2)$,CPU处理缓慢。通过GPU并行化像素循环,可显著加速。

\subsection{代码实现与优化}
以下是伪代码:
\begin{algorithm}[H]
\caption{Back-Projection Algorithm (BPA)}
\begin{algorithmic}[1]
\State \textbf{Input:} Raw Data $D \in \mathbb{C}^{N_a \times N_r}$, Coordinates $Grid \in \mathbb{R}^{N_{img} \times N_{img}}$
\State \textbf{Output:} Image $I$
\State \cpuonly{Initialize image $I = 0$} \gpuacc{Allocate GPU memory}
\For{each pulse $\eta = 1$ to $N_a$}
    \For{each pixel $(x,y)$ in Grid \gpuacc{(parallel)}}
        \State Compute range $R = \sqrt{(x - x_\eta)^2 + (y - y_\eta)^2 + R_0^2}$
        \State Delay $\tau = 2R/c$
        \State Interpolate $D(\eta, \tau)$
        \State Phase compensate $\exp(j 4\pi R / \lambda)$
        \State Accumulate to $I(x,y)$
    \EndFor
\EndFor
\end{algorithmic}
\end{algorithm}
\complexity{N_a N_r N_{img}^2}{N_a N_r N_{img}^2 / threads}

Python实现使用CuPy加速(关键片段):
\begin{lstlisting}
import cupy as cp

def bpa_gpu(raw_data, grid_x, grid_y, pos_a, lambda_, c):
    raw_gpu = cp.array(raw_data)
    img = cp.zeros((len(grid_x), len(grid_y)), dtype=cp.complex64)
    
    for eta in range(raw_data.shape[0]):
        dx = grid_x - pos_a[eta, 0]
        dy = grid_y - pos_a[eta, 1]
        R = cp.sqrt(dx**2 + dy**2 + R0**2)
        tau_idx = cp.round(2 * R / c * fs).astype(cp.int32)
        phase = cp.exp(1j * 4 * cp.pi * R / lambda_)
        valid = (tau_idx >= 0) & (tau_idx < raw_data.shape[1])
        img[valid] += raw_gpu[eta, tau_idx[valid]] * phase[valid]
    
    return cp.asnumpy(img)
\end{lstlisting}

实验配置:Tesla T4 GPU,数据规模128x128到1024x1024。

\subsection{性能对比}
执行时间对比(单位:秒):
\begin{table}[H]
\centering
\begin{tabular}{lccc}
\toprule
算法 & CPU时间 & GPU时间 & 加速比 \\
\midrule
BPA & 10.253 & 0.079 & 129.8 \\
CSA & 1.330 & 0.033 & 40.3 \\
RDA & 1.793 & 0.942 & 1.9 \\
wKA & 1.369 & 1.120 & 1.2 \\
\bottomrule
\end{tabular}
\caption{GPU vs CPU 执行时间对比}
\label{tab:gpu_perf}
\end{table}

\begin{figure}[H]
\centering
\includegraphics[width=0.8\linewidth]{cpu_gpu_output.png}
\caption{GPU vs CPU 执行时间条形图}
\label{fig:gpu_comparison}
\end{figure}



\section{其他算法对比:CSA, RDA, wKA}
- CSA:频域算法,处理距离徙动修正(RCMC),适用于线性轨迹。
- RDA:结合距离多普勒,高效但需理想假设。
- wKA:Omega-K算法,频域精确,修正频率依赖。

理想条件下,所有算法产生类似图像,但BPA精度最高。

\begin{figure}[H]
\centering
\includegraphics[width=0.8\linewidth]{cpu_gpu.png}
\caption{GPU vs CPU 算法效率对比}
\label{fig:gpu_comparison_perf}
\end{figure}

上面的BPA算法和wKA算法有着明显的条带,这实际上就是与点扩散函数有关系,或者也可以理解为BPA是沿着距离徙动曲线执行的,但是在信噪比稍微小些的条件下他们会运行的更好

回波信号就生成就是使用课件的方法,这里不再赘述,并且生成经典的图像的时候,就是4000*1400的时候GPU比CPU快了30倍左右
\begin{figure}[H]
\centering
\includegraphics[width=0.75\linewidth]{echo_data.png}
\caption{回波信号生成}
\label{fig:echo_generation}
\end{figure}

\section{噪声影响分析}
在无噪声理想状态下,成像清晰,但缺少噪声导致过完美相干积累,形成X形波带。这是因为距离徙动曲线(RCM)在无噪声时完美叠加。

实际中,添加噪声可模拟真实场景(见附录代码)。

\section{成像质量分析}

\begin{table}[H]
\centering
\caption{各算法成像质量对比(中心目标)}
\label{tab:imaging_quality}
\begin{tabular}{|c|c|c|c|c|c|}
\hline
\textbf{算法} & \textbf{距离分辨率 (m)} & \textbf{理论偏差 (\%)} & \textbf{方位分辨率 (m)} & \textbf{理论偏差 (\%)} & \textbf{PSLR (dB)} \\ \hline
BPA            & 1.88                  & +0.3\%                 & 0.142                   & +0.4\%                 & -13.2              \\ \hline
RDA            & 1.94                  & +3.5\%                 & 0.145                   & +2.5\%                 & -12.9              \\ \hline
CSA            & 1.92                  & +2.5\%                 & 0.144                   & +1.8\%                 & -13.1              \\ \hline
wKA (修复)      & 1.89                  & +0.9\%                 & 0.143                   & +1.1\%                 & -13.3              \\ \hline
\end{tabular}
\end{table}

\noindent 从表\ref{tab:imaging_quality}可以看出:
\begin{itemize}
    \item \textbf{理论分辨率:}距离=1.874 m,方位=0.1415 m
    \item \textbf{BPA算法}表现出最佳的成像质量,距离和方位分辨率非常接近理论值(1.88 m和0.142 m,偏差均小于0.5\%)
    \item \textbf{RDA算法}距离分辨率略有拓宽(1.94 m,偏差+3.5\%),这是由于RCMC插值带来的误差
    \item \textbf{CSA算法}通过Chirp Scaling避免了RCMC插值,成像质量优于RDA(1.92 m和0.144 m,偏差均在3\%以内)
    \item \textbf{wKA算法(修复版)}在修正了Stolt映射和相位补偿后,分辨率显著提升(1.89 m和0.143 m,偏差均小于2\%),达到了与BPA相当的精度
    \item \textbf{PSLR指标:}所有算法的PSLR均在-12.9 dB至-13.3 dB之间,接近Sinc函数的理想值(-13.26 dB)
\end{itemize}

\noindent \textbf{综合评估:}
\begin{itemize}
    \item \textbf{BPA}作为精确时域算法,成像质量最佳,距离和方位分辨率偏差均小于0.5\%,但计算量最大
    \item \textbf{RDA}和\textbf{CSA}作为频域算法,实现了计算效率与成像质量的良好平衡,偏差均在5\%以内
    \item \textbf{wKA(修复版)}在保持较高成像质量的同时(偏差小于2\%),计算效率也较为理想
    \item 所有算法的实测分辨率与理论值偏差均在5\%以内,验证了仿真系统的可靠性
\end{itemize}


具体的一些图像如下
\begin{figure}[H]
\centering
\includegraphics[width=0.75\linewidth]{psf_rda.png}
\caption{RDA图像}
\label{fig:psf_rda}
\end{figure}

\begin{figure}[H]
\centering
\includegraphics[width=0.75\linewidth]{psf_csa.png}
\caption{CSA图像}
\label{fig:psf_csa}
\end{figure}

\begin{figure}[H]
\centering
\includegraphics[width=0.75\linewidth]{psf_wka.png}
\caption{wkA图像}
\label{fig:psf_wka}
\end{figure}

\noindent \textbf{说明:}
\begin{itemize}
    \item 理论分辨率:距离=1.874 m,方位=0.1415 m
    \item 修复版的wKA算法解决了原始实现中目标漂移和主瓣拓宽的问题。
    \item PSLR(峰值旁瓣比)接近理想理论值-13.26dB。
\end{itemize}

\section{小结}
本章实现了SAR成像算法,验证了GPU加速的有效性。理想条件下,BPA提供最佳质量,但需注意噪声缺失的影响。下一章将引入Hermite函数扩展抗噪能力。
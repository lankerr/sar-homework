\chapter{分数阶傅里叶变换的关系}

\section{引言与研究背景}
在本章中,我们探讨分数阶傅里叶变换(Fractional Fourier Transform, FRFT)与合成孔径雷达(SAR)成像的关系。SAR的高分辨率成像源于其信号处理的时频特性,特别是线性调频(LFM)信号的模糊函数呈现刀刃状,这表明LFM信号在时频平面旋转特定角度后可压缩为冲激函数。Chirp Scaling Algorithm(CSA)等去斜脉冲压缩算法本质上处理这类时频信号。

这种特性对应于Hermite函数系,该系在傅里叶变换下保持不变,具有旋转不变性。我们将LFM信号分解为Hermite基函数的加权和,并引入稀疏Hermite滤波以提升抗干扰能力。研究过程包括数学推导、系统设计和实验验证。

\section{Hermite函数系与SAR时频对称性}
Hermite函数系是时频分析的基础,其时域和频域形式对称:
\begin{equation}
\exp(-j\omega^2) \longleftrightarrow \exp(-t^2)
\end{equation}

Hermite多项式定义为:
\begin{equation}
H_n(t) = (-1)^n e^{t^2} \frac{d^n}{dt^n} e^{-t^2}
\end{equation}

标准化Hermite函数:
\begin{equation}
h_n(t) = \frac{1}{\sqrt{2^n n! \sqrt{\pi}}} e^{-t^2/2} H_n(t)
\end{equation}

这些函数构成正交基,且在傅里叶变换下自闭合。SAR中的LFM信号可视为Hermite系的线性组合,其时频响应为圆形,确保旋转不变。

\section{LFM信号的Hermite分解}
任何chirp信号可表示为Hermite基的加权和:
\begin{equation}
s(t) = \sum_{n=0}^\infty c_n h_n(t)
\end{equation}
其中系数$c_n = \int s(t) h_n(t) dt$。

实验显示,LFM信号能量分布在低阶Hermite上,而稀疏Hermite信号集中在特定阶上。这为抗干扰提供基础:LFM干扰能量分散,易于滤除。

\begin{figure}[H]
\centering
\includegraphics[width=0.8\linewidth]{LFM_hermite.png}
\caption{LFM信号在Hermite域的能量分布}
\label{fig:lfm_hermite}
\end{figure}

\textbf{实际上经过mathematical计算可以得到解析式:}

高斯窗 Chirp 信号在 Hermite 基底下的投影系数 $c_n$ 的精确数学表达式。设信号为 $s(t) = e^{-\alpha t^2} e^{j \pi \beta t^2}$,复参数 $a = \alpha + \frac{1}{2} - j \pi \beta$。投影系数 $c_n$ 的解析式为:当 $n$ 为奇数时:$$c_n = 0$$当 $n$ 为偶数时 ($n=2m$):$$c_n = \underbrace{\frac{1}{\sqrt{2^n n! \sqrt{\pi}}}}_{\text{归一化系数}} \times \underbrace{\sqrt{\frac{\pi}{a}}}_{\text{积分项}} \times \underbrace{\frac{n!}{(n/2)!}}_{\text{阶乘项}} \times \underbrace{\left( \frac{1}{a} - 1 \right)^{n/2}}_{\text{衰减/振荡项}}$$
我们可以看到
$$||c_n||^2 \text{受} \frac{1}{2^nn!} \text{主导}$$

\section{FrFT一种时频变换的方法}
分数阶傅里叶变换(Fractional Fourier Transform, FrFT)可以被视为传统傅里叶变换的广义推广。从时频平面的角度来看,普通傅里叶变换相当于将信号在时频平面上旋转 $\pi/2$ 弧度。而 FrFT 则引入了一个连续的旋转角度参数 $\phi = \alpha \pi / 2$,使得我们可以在任意角度观察信号的投影。

从 Hermite 正交基底的角度来看,FrFT 的定义非常自然:由于标准化 Hermite 函数 $h_n(t)$ 是傅里叶变换的特征函数,其对应的特征值为 $e^{-j n \pi / 2}$。因此,FrFT 可以定义为在 Hermite 分解空间中引入一个与阶数 $n$ 和旋转因子 $\alpha$ 相关的相位因子:
\begin{equation}
\mathcal{F}^\alpha[s(t)] = \sum_{n=0}^\infty c_n e^{-j n \alpha \pi / 2} h_n(t)
\end{equation}

其主要数学特性包括:
\begin{enumerate}
    \item \textbf{线性性}:满足线性叠加原理,方便在复杂信号环境下处理。
    \item \textbf{可加性}:$\mathcal{F}^\alpha \mathcal{F}^\beta = \mathcal{F}^{\alpha+\beta}$,即连续两次旋转的效果等于一次角度累加的旋转。
    \item \textbf{旋转可加性}:直接对应于时频平面的几何旋转。
    \item \textbf{Hermite 正交基底}:作为变换的固有基,保证了在旋转变换过程中的信号能量守恒。
\end{enumerate}

在时域中,该变换可以通过以下积分核公式实现:
\begin{equation}
X_\alpha(u) = \int_{-\infty}^{\infty} s(t) K_\alpha(t, u) dt
\end{equation}
其中,$K_\alpha(t, u)$ 为:
\begin{equation}
K_\alpha(t, u) = \sqrt{\frac{1-j \cot \phi}{2\pi}} \exp\left(j \frac{t^2+u^2}{2} \cot \phi - j t u \csc \phi\right)
\end{equation}

\section{抗干扰系统设计(稀疏Hermite滤波)}
提出基于稀疏Hermite的抗干扰系统,不依赖干扰参数$K_r$估计,利用Hermite的时频对称性抑制LFM干扰。

系统流程:
1. 将接收信号投影到Hermite基。
2. 保留稀疏阶(已知信号阶),滤除其他(干扰主导)。
3. 重构信号。

\begin{figure}[H]
\centering
\includegraphics[width=0.8\linewidth]{sparse_compress.png}
\caption{稀疏之后的重构}
\label{fig:stripe_diagnosis}
\end{figure}
使用稀疏滤波器我们可以看到,几乎被淹没的信号被重建出来了,这得益于其模糊函数是一个圆环,任何噪声都很难将其各个方向进行干扰

量化指标:干扰抑制比(JSR):
\begin{equation}
\text{JSR} = 10 \log_{10} \left( \frac{\sum_{n \notin \mathcal{H}} |d_n|^2}{\sum_{n \in \mathcal{H}} |d_n|^2} \right)
\end{equation}

实验中JSR达20-30 dB。

\section{实验验证}
使用模拟数据验证:LFM系统干扰产生假目标;Hermite系统正交抑制;稀疏Hermite进一步提升SCR。


额外图像验证:

在二维情形在0处我们放置了的真值,但是在-5处 we 使用了超强干扰
\begin{figure}[H]
\centering
\includegraphics[width=0.8\linewidth]{hermite_anti.png}
\caption{干扰虚假目标}
\label{fig:hermite_anti_jam}
\end{figure}
hermitenaive仅仅使用一个hermite基底,抗干扰能力不能被充分发挥,但是如果利用其稀疏性我们可以看到是可以看到尖峰是在0处的,-5处的干扰被完全的抑制,提升达到27dB

\begin{table}[H]
\centering
\begin{tabular}{lcc}
\toprule
系统 & 干扰/信号比 & SCR提升 (dB) \\
\midrule
LFM & 1.0000 & 0.0 \\
Hermite & 0.0010 & 30.0 \\
Sparse Hermite & 0.000001 & 60.0 \\
\bottomrule
\end{tabular}
\caption{抗干扰性能对比}
\label{tab:antijam_perf}
\end{table}

\section{基于FrFT的抗干扰}
前面我们重点讲解了利用高阶 Hermite 基底的稀疏性来抗干扰的方法,其核心逻辑在于高阶 Hermite 函数与 LFM 干扰在某种程度上具有“天然”的不相关性。然而,我们还可以从另一个更具几何直观性的角度来处理干扰:分数阶傅里叶变换。

线性调频(LFM)信号在时频平面上表现为一条倾斜的直线。当我们将观察视角通过 FrFT 旋转到与该直线垂直的角度 $\alpha_{opt}$ 时,原本发散的 LFM 干扰会坍缩成一个极窄的峰值,其数学性质几乎接近于 Delta 函数。

\begin{figure}[H]
\centering
\includegraphics[width=0.8\linewidth]{frft_angle_search.png}
\caption{LFM 信号在 FRFT 域的集中度随旋转角度的变化}
\label{fig:frft_angle_search}
\end{figure}

如图 \ref{fig:frft_angle_search} 所示,通过搜索能量集中度最高的角度,我们可以精确锁定 LFM 干扰所在的“焦点”。在这种状态下,我们只需要在分数域插入一个窄带陷波(Notch)滤波器,就可以极其高效地滤除大部分干扰能量,而对有用信号的损伤降到最低。

更进一步,我们可以将 FrFT 旋转删除与 Hermite 稀疏重构结合起来。如图 \ref{fig:frft_antijam_summary} 所示,组合方法(Combined)相比于单一的 Hermite 滤波或单纯的 FRFT 去峰,能显著提升恢复信号的信噪比(SNR)。

\begin{figure}[H]
\centering
\includegraphics[width=0.9\linewidth]{frft_antijam_summary.png}
\caption{FRFT 与 Hermite 联合抗干扰流程与性能对比}
\label{fig:frft_antijam_summary}
\end{figure}

这种“旋转+稀疏”的联合策略充分利用了信号在不同变换域的分布差异:FrFT 负责将强干扰“聚焦”并剔除,而 Hermite 域则负责在干扰滤除后的残差中,利用信号的先验稀疏性进行精准重构。

\begin{figure}[H]
\centering
\includegraphics[width=0.8\linewidth]{antijam_time_comparison.png}
\caption{时域恢复效果对比:受干扰信号 vs 联合算法恢复信号}
\label{fig:antijam_time_comparison}
\end{figure}

从时域恢复结果(图 \ref{fig:antijam_time_comparison})可以看出,即便 JSR 较高,联合算法依然能够完美还原原始 Hermite 信号的波形特征,验证了该方案在极复杂干扰环境下的稳定性。

\section{小结}
Hermite函数系与分数阶傅里叶变换(FrFT)为SAR抗干扰提供了坚实的理论基础。本章通过对LFM信号的Hermite分解及其在分数域的旋转特性分析,提出并验证了稀疏滤波与旋转删除的联合抑制算法。实验结果表明,该方案能有效抑制强LFM干扰并精准恢复有用信号。下一章将这些时频处理方法扩展到二维成像与BPA算法的融合中。